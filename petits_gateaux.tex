\documentclass{recettes}

\title{Petits gâteaux}
\author{Famille Idir}

\begin{document}
	
\maketitle

\recette{Shortbread (Écosse)}
Pour 80 à 100 pièces:
\begin{ingredients}
	\item 450 g farine (300 g brune et 150 g blanche)
	\item 200 g beurre + margarine
	\item 250 g sucre de canne
\end{ingredients}
  Dans un saladier : beurre mou + sucre + farine + un peu d'eau si besoin.\\ Former une boule à laisser 1/2h au frigo.\\
En utilisant environ 1/4 de la préparation initiale à chaque fois, étaler la pâte pour obtenir une plaque de 2 cm d'épaisseur. La découper en pièces de 2 cm de coté.\\
Recouvrir d'un papier de cuisson une plaque de four, y disposer les pièces dessus. A l'aide du pouce les écraser légèrement en leur centre.\\
Cuisson: 25 min th 180°C, ces gâteaux durcissent en séchant.


\recette{Croissants vanille (Allemagne)}
Pour 50 pièces:
\begin{ingredients}
	\item 210 g farine
	\item 50 g sucre
	\item 140 g beurre
	\item 3 paquets sucre vanillé
	\item 70 g noisettes en morceaux
	\item 2 CS sucre glace
\end{ingredients}
Beurre + noisettes + sucre + farine. Laisser reposer 1h à 4°.\\
Travailler la pâte de manière à obtenir de petits croissants.\\
Les disposer sur un papier de cuisson.\\
Cuisson th 175°, 25 min pour la première fournée et que 20 pour les suivantes.\\
Préparer le mélange de sucre vanille et sucre glace. Y rouler les gâteaux chauds.\\

\recette{Macarons au chocolat}
\begin{ingredients}
\item 4 blancs d'œufs montés en neige
\item 125 g chocolat amer
\item 250 g sucre en poudre
\item 250 amandes en poudre
\item hosties ou papier pour four
\end{ingredients}
Mélanger les ingrédients et en faire des petits tas sur les hosties.
180° pendant 15 min avec un four préchauffé. En fin de cuisson, le macaron doit être juste craquant en surface et mou à l'intérieur.

\recette{Triangle aux noix:}
Préparation: 30 min	

\begin{ingredients}
\item 150 g beurre (ou margarine)
\item 150 g sucre
\item 190 g noix moulues
\item 250 g farine
\item 1 CS rhum
\item 1 œuf
\end{ingredients}

  Battre le beurre fondu avec le sucre. Ajouter les autres ingrédients avec 2 CS d'eau. Former des prismes d'environ 2 cm de coté, et les découper en tranches d'1 cm.\\
  Cuisson 20 min à 190°C.

\recette{Nids aux noisettes}
\image{0.7}{nids_aux_noisettes.png}
Préparation: 15 min
Pour 65 pièces:

\begin{ingredients}
\item 150 g sucre
\item 100 g beurre (ou margarine) 
\item 2 œufs + 1 jaune
\item 300 g farine
\item 1/2 paquet de levure
\item     noisettes entières
\end{ingredients}

  Tourner en mousse le sucre, les œufs et le beurre. \\
  Ajouter en pluie la farine et la levure préalablement mélangées.\\
  Placer sur une plaque de four des petites boulettes de pâte (2 cm de diamètre) où une noisette est plantée au centre.\\
  Cuisson 15 min à 180°C.



\recette{Galettes à l'orange:}
Préparation: 20 min + 1h de repos
Cuisson: 12 min

Pour 40 pièces:	
\begin{ingredients}					
\item 120 g beurre (ou margarine)				
\item 2 jaunes d'œufs					  
\item 80 g sucre
\item 1/2 jus d'orange + zeste
\item 1 pointe de couteau de levure chimique
\item 250 g farine
\end{ingredients}
\textbf{Glaçage:}
\begin{ingredients}
\item 150 g sucre glace
\item 1 CS jus d'orange
\end{ingredients}

  Blanchir le mélange sucre + beurre. Ajouter les jaunes d'œuf, le jus d'orange et le zeste, puis + farine et levure. Former une boule mise a reposer 1h au frigo.\\
  Faire une abaisse de 5 mm et découper des disques (5 cm de diamètre). \\
  Cuisson 12 min à 200°.\\
  Glacer encore chaud.  

\recette{Petits rochers à la noix de coco:}
\image{0.5}{rochers_coco.png}
  Pour 30 pièces :
  \begin{ingredients}
  \item 200 g noix de coco râpée
  \item 120 g sucre poudre (sucre déduit)
	\item 2 œufs
   \item 40 g beurre
  \end{ingredients}
  
	Noix de coco + sucre + beurre. Faire fondre le beurre 30 secondes au micro-onde.\\
  + 2 œufs, mélanger rapidement.\\
	 Disposer de petits tas façonnés en forme de pyramide sur une plaque de cuisson. Cuisson 10 min à 180°.


\recette{Croquants aux noix}
\image{0.5}{croquants_noix.png}
\begin{ingredients}
\item 1 tasse de noix moulues
\item 1/2 tasse de crème fraîche
\item 3/4 tasse de sucre (sucre réduit)
\item 3/4 tasse de farine
\item 1 œuf
\item cannelle
\end{ingredients}

Cuisson environ 15 min à 180°C.

\recette{Mini-cakes}
Préparation : 20 min
Cuisson : 2*15 min

Pour 48 pièces :
\begin{ingredients}
\item 190 g farine
\item 190 g sucre
\item 150 g beurre
\item 4 œufs
\item 1/2 sachet de levure
\item sel
\end{ingredients}

  Dans un saladier, mélanger le sucre avec les œufs et le sel pour obtenir une préparation mousseuse. + farine, + levure + beurre ramolli.\\
  Préchauffer le four à 180°C.\\
  Beurrer la plaque a mini-cakes et verser-y la pâte en remplissant entre la moitié et les 3/4.\\
  Cuisson 15-20 min. Démouler à froid.\\
\\
Variante : ces gâteaux peuvent être parfumés avec zeste d’orange, de citron, pépites de chocolats, raisins secs ...

\recette{Croquets}
\image{0.5}{croquets.png}
Pour la préparation de 2 pains lors de la cuisson :
\begin{ingredients}
\item 500 g farine
\item 4 œufs
\item 250 g sucre poudre
\item 70 g huile
\item 1 sachet de levure chimique
\item 1 CS rhum
\item 200 g amandes ou noisettes entières
\end{ingredients}

  Mélanger le tout en n’ajoutant pas la levure en dernier afin de mieux la répartir.\\
  Constituer 2 pains de 1 à 2 cm d’épaisseur. Les disposer sur une plaque à four farinée.\\
  Cuisson 40 min à 180°C.\\
  A la sortie du four, découper les pains en tranches de 1 cm.\\
  Laisser refroidir. 

\recette{Petits beurre aux fruits confits :}
\image{0.5}{Petits_beurre_aux_fruits_confits.png}
Pour 40-50 biscuits :
\begin{ingredients}
\item 180 g beurre
\item 200 g cassonade
\item 100 g fruits confits
\item 400 g farine
\item 1 sachet sucre vanillé 
\item 1 œuf
\end{ingredients}

  Cassonade + sucre vanillé + beurre battu en crème + œuf puis farine et fruits confits.\\
  Former 1 ou 2 rouleaux de 3 cm de diamètre. \\
Envelopper de film plastique ou de papier sulfurisé pour les mettre au moins 30 min à 4°C.\\
  Couper les rouleaux en tranches de 0.5 cm et les disposer sur du papier sulfurisé.\\
  Cuisson 15 min à 180°C.\\

Variante : Remplacer les fruits confits par 100 à 200 g de raisins secs.

\recette{Croquants aux cacahouètes}
\image{0.5}{Croquants_cacahouètes.png}
\begin{ingredients}
\item 100 g cacahouètes moulues
\item 50 g farine
\item 120 g sucre
\item 140 g crème fraîche
\item 1 œuf
\item 5 CS lait
\end{ingredients}

  Mélanger tout les ingrédients. Remplir à ras les moules rectangulaires en silicone. Cuisson 20-25 min à 180°C.

\recette{Petits beurres}
  Pour 30 biscuits
\begin{ingredients}
  \item 125 g beurre
  \item 100 g sucre
  \item 1 sachet sucre vanillé
  \item 200 g farine
  \item 1 œuf
  \item 1,5 CC levure
\end{ingredients}
  
	Beurre + sucres + œuf, puis + farine et levure.\\
	Déposer 1 CC de pâte sur la plaque de cuisson.\\
	Cuisson 15 min à 180°C.\\
Se garde bien.

\recette{Biscuits chinois}
\begin{ingredients}
\item 125 g beurre ramolli
\item 80 g sucre
\item 240 g farine
\item 1 jaune d’œuf
\item 1/2 CC levure chimique
\item 1 CC essence d’amande
\item 15-20 amandes blanchies
\end{ingredients}
 
  Battre le beurre en crème avec le sucre + essence d’amande et jaune d’œuf, puis + farine. Mélanger jusqu’à obtention d’une pâte ferme. Former un rouleau et couvrir d’un film plastique. Mettre au frais durant 30 min.\\
  Couper la pâte en tranche de 0,25 cm. Déposer à plat sur la plaque silicone, et déposer une demie amande au centre, et l’y enfoncer.\\
Cuisson 10-15 min à 180°C.\\
Ces biscuits ne se développent pas à la cuisson.\\
Se conservent bien.\\

Variante : + 40g de graines de sésame

\recette{Biscuits de Pâques (ss beurre ni œuf)}
Pour 30-40 biscuits
\begin{ingredients}
  \item 300 g farine
  \item 100 g sucre
  \item 80 g huile
  \item  3 CS vin blanc (facultatif)
  \item  3 CS pastis
\end{ingredients}

  Farine + sucre puis + huile, vin, alcool. Au besoin + eau. Pétrir jusqu’à ce que la pâte soit lisse.\\
  Abaisser la pâte de 0.5 cm d’épaisseur. La découper à l’emporte pièce.\\
  Décoration possible avec 1 raisin sec appuyé au centre.\\
  Ne se développe pas au four. Cuisson 15 min à 180°C.\\
  Se conserve bien.

\recette{Croquants à l’avoine}
\image{0.5}{croquants_avoine.png}
  Pour 24-30 biscuits : 
  \begin{ingredients}
    \item   160 g farine
    \item  100 g beurre
    \item   60 g sucre en poudre (sucre réduit)
    \item   60 g cassonade (sucre réduit)
    \item  1 oeuf
    \item  2 CS lait
    \item  1 sachet sucre vanillé
    \item  150 g flocons d’avoine
    \item  1 pointe de cannelle
    \item  1 pointe de muscade
    \item  1/2 CC levure chimique
  \end{ingredients}
  
    Beurre mou + les divers sucres, puis + oeufs, lait, épices, pincée de sel puis le reste des ingrédients.\\
    Déposer 1 CC de pâte par loge d’un moule téflon.\\
    Cuisson 15 min à 180°C.\\
  
    Décoration possible par de petits carrés de dates avant cuisson.

\recette{Sablés nature}
\image{0.5}{sablés_nature.png}
    Pour plus de 60 pièces à l’emporte pièce
  \begin{ingredients}
  \item 350 g farine
  \item 200 g beurre très ramolli
  \item 120 g sucre
  \item 1 sachet sucre vanillé
  \item 1 œuf
  \item zeste d’un citron non traité (facultatif)
  \end{ingredients}
    
  Découper la pâte étalée avec un emporte pièce. Dans le cas des biscuits miroirs, la pâte doit être très fine.
    
    Cuisson 15 min à 180°C.

\recette{Petits pains d’épice tchèque}
\image{0.5}{pain_épices_tchèque.png}
    Pour 60 sujets à l’emporte-pièce
    \begin{ingredients}
    \item 165 g farine
    \item 70 g sucre (mieux si sucre glace)
    \item 1 œuf
    \item 10 g beurre
    \item 1/3 CC bicarbonate de soude
    \item 1/3 CC cacao
    \item 1 pincée de sel
    \item vanille, anis
    \end{ingredients}
    
Tout mélanger. Laisser reposer plusieurs heures.\\
Etaler en 5 mm d’épaisseur, et découper à l’emporte-pièce.\\
Cuisson 10 min à 180°C.

\recette{Langues de chat}
\begin{ingredients}
\item 60 g farine
\item 60 g sucre en poudre
\item 2 blancs d’œuf
\item 60 g beurre
\item 2 gouttes d’essence de vanille
\end{ingredients}

  Beurre crémeux + sucre + vanille + farine. Bien mélanger. \\
  + blancs d’œufs montés en neige.\\
  Beurrer une plaque de four, étaler la pâte en bâtonnets espacés.\\
  Cuisson 7-8 min à 200°C.

  \recette{Biscuit citronné au yaourt}
  Pour 30 à 40 biscuits :
  \begin{ingredients}
  \item 300 g farine
  \item 50 g maïzéna
  \item 200 g sucre en poudre
  \item 2 œufs
  \item 150 g beurre ramolli
  \item 1 CS zeste de citron (facultatif)
  \item 3 CC jus de citron
  \item 2 CS yaourt
  \end{ingredients}
  
Beurre crémeux + sucre battus en crème. + citron (zeste et jus) + œufs puis + farine et yaourt. Bien mélanger. \\
+ blancs d’œufs montés en neige.\\
Sur du papier sulfurisé, disposer 1 CC de pâte qui s’étalera à la cuisson.  \\
Cuisson 15-20 min à 180°C.


\recette{Feuillantines aux amandes}
\image{0.5}{feuillantines_amandes.png}
Pour   biscuits :
\begin{ingredients}
\item 250 g farine
\item 180 g sucre cassonade
\item 1 œuf
\item 125 g beurre ramolli 
\item 125 g amandes mondées entières 
\item 1/2 CC cannelle
\item 1/2 CC levure chimique
\item 1 pincée de sel
\end{ingredients}
  
  Mélanger tous les ingrédients sauf les amandes.\\
  Former 1 ou 2 blocs de section rectangulaire (3x4,5 environ).\\
  Presser les amandes dans la pâte et laisser reposer toute la nuit au frais. \\
  Découper des tranches de 1-2 mm d’épaisseur.\\
  Cuisson 10-12 min à 200°C dans le tiers supérieur du four préchauffé.

\recette{Caramellini}
\image{0.5}{caramellini.png}
  Pour  \~ 80 biscuits : 
  \begin{ingredients}
  \item 250 g farine
  \item 170 g sucre cassonade
  \item 1 œuf
  \item 100 g beurre ramolli 
  \item 1 CC sucre vanillé
  \item 1 CS miel
  \item 1 CC levure chimique
  \item 2 pincées de sel
  \end{ingredients}
    
Travailler en mousse le beurre avec la cassonade. Ajouter les autres ingrédients avec la levure chimique en dernier. Bien mélanger avant et après l’apport de levure.\\
Former des boudins de 2-3 cm de diamètre. Laisser durcir au frigidaire.\\
Couper en rondelles de 3 mm de diamètre et les disposer sur une plaque de cuisson non graissée (s’étalent un peu à la cuisson)\\
Cuisson 6 à 8 min à 210°C dans la moitié supérieure du four.

\recette{Bredele rhum chocolat}
\image{0.5}{bredele_chocolat.png}
Pour une trentaine de pièces
\begin{ingredients}
 \item 2 œufs
 \item 200 g sucre en poudre 
 \item 2 sachets sucre vanillé
 \item 1 CS rhum
 \item 75 g chocolat pâtissier en poudre
 \item 150 g noisettes en poudre
 \item 30 g fécule (maïzena)
 \end{ingredients}

Mélanger l’ensemble des ingrédients, et disposer dans un moule téflon à fond sphérique. Cuisson 15 min à 180°C.

\recette{Boules à la noix de coco}
\image{1}{boules_coco.jpg}
Pour 50 pièces :
\begin{ingredients}
\item 180 g beurre
\item 100 g sucre
\item 1 sachet sucre vanillé
\item 1 zeste râpé d'une orange
\item 1 pincée sel
\item 1 blanc d'œuf
\item 250 g farine (si besoin + 50g)
\item 100 g amandes en poudre
\item 200 g noix de coco râpée
\item 5 CS rhum
\end{ingredients}

  Battre en mousse beurre + les 2 sucres, + zeste d'orange, sel et blanc d'œuf, + le reste des ingrédients. Pétrir jusqu'à l'obtention d'une pâte homogène. Former des boules de la taille d'une grosse noisette.\\
  Cuisson à 170°C (chaleur tournante) pendant 20 min.\\
Dès la sortie du four, rouler les boules dans le mélange sucre glace et sucre vanillé puis laisser refroidir.

\recette{Etoiles à la cannelle revisitées}
\image{0.5}{etoiles_cannelle.png}
Pour 30 pièces :
\begin{ingredients}
\item 200 g sucre glace
\item 3 blancs d'œufs
\item 50 g farine (si besoin + 50g)
\item 350 g amandes en poudre
\item 1/2 CS cannelle
\end{ingredients}

Battre les blancs en neige très ferme. Puis + le reste des ingrédients.\\
Disposer une 1/2 CC dans les loges d’une plaque de moule silicone (en forme mini-brioche). Cuisson à 5-10 min à 180°C (ou 210°C).

\recette{Navettes de Mamie}
\begin{ingredients}
  \item 1 bol de crème fraiche épaisse 40\% matière grasse
  \item 1/2 bol de sucre
  \item 2 œufs
  \item 1 pincée de sel
  \item Farine jusqu’à obtenir une pâte qui se tient.
\end{ingredients}

Faire un boudin aplati et le couper en tranches.\\
Cuisson 20 min à 120-150°C.

\recette{Cookies (recette de Marc)}
Pour 45 biscuits : 
\begin{ingredients}
\item 250 g farine
\item 60 g sucre cassonade
\item 60 g sucre blanc
\item 2 œufs
\item 125 g beurre ramolli 
\item 60 g chocolat noir en petits morceaux
\item 30 g noisettes en morceaux
\end{ingredients}
  
  Mélanger au robot tous les ingrédients sauf chocolat et noisettes rajoutés au dernier moment. \\
  Répartir en petits tas sur une plaque de cuisson. S’étalent très légèrement à la cuisson.\\
  Cuisson entre 12 et 14 minutes à 190°C.

\end{document}
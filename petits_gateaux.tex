\documentclass{recettes}

\title{Petits gâteaux}

\begin{document}
	
\maketitle

\recette{Shortbread (Écosse)}
Pour 80 à 100 pièces:
\begin{ingredients}
	\item 450 g farine (300 g brune et 150 g blanche)
	\item 200 g beurre + margarine
	\item 250 g sucre de canne
\end{ingredients}
  Dans un saladier : beurre mou + sucre + farine + un peu d'eau si besoin.\\ Former une boule à laisser 1/2h au frigo.\\
En utilisant environ 1/4 de la préparation initiale à chaque fois, étaler la pâte pour obtenir une plaque de 2 cm d'épaisseur. La découper en pièces de 2 cm de coté.\\
Recouvrir d'un papier de cuisson une plaque de four, y disposer les pièces dessus. A l'aide du pouce les écraser légèrement en leur centre.\\
Cuisson: 25 min th 180°C, ces gâteaux durcissent en séchant.


\recette{Croissants vanille (Allemagne)}
Pour 50 pièces:
\begin{ingredients}
	\item 210 g farine
	\item 50 g sucre
	\item 140 g beurre
	\item 3 paquets sucre vanillé
	\item 70 g noisettes en morceaux
	\item 2 CS sucre glace
\end{ingredients}
Beurre + noisettes + sucre + farine. Laisser reposer 1h à 4°.\\
Travailler la pâte de manière à obtenir de petits croissants.\\
Les disposer sur un papier de cuisson.\\
Cuisson th 175°, 25 min pour la première fournée et que 20 pour les suivantes.\\
Préparer le mélange de sucre vanille et sucre glace. Y rouler les gâteaux chauds.\\





\end{document}
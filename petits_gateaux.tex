\documentclass{recettes}

\title{Petits gâteaux}
\author{Famille Idir}

\begin{document}
	
\maketitle

\recette{Shortbread (Écosse)}
Pour 80 à 100 pièces:
\begin{ingredients}
	\item 450 g farine (300 g brune et 150 g blanche)
	\item 200 g beurre + margarine
	\item 250 g sucre de canne
\end{ingredients}
  Dans un saladier : beurre mou + sucre + farine + un peu d'eau si besoin.\\ Former une boule à laisser 1/2h au frigo.\\
En utilisant environ 1/4 de la préparation initiale à chaque fois, étaler la pâte pour obtenir une plaque de 2 cm d'épaisseur. La découper en pièces de 2 cm de coté.\\
Recouvrir d'un papier de cuisson une plaque de four, y disposer les pièces dessus. A l'aide du pouce les écraser légèrement en leur centre.\\
Cuisson: 25 min th 180°C, ces gâteaux durcissent en séchant.


\recette{Croissants vanille (Allemagne)}
Pour 50 pièces:
\begin{ingredients}
	\item 210 g farine
	\item 50 g sucre
	\item 140 g beurre
	\item 3 paquets sucre vanillé
	\item 70 g noisettes en morceaux
	\item 2 CS sucre glace
\end{ingredients}
Beurre + noisettes + sucre + farine. Laisser reposer 1h à 4°.\\
Travailler la pâte de manière à obtenir de petits croissants.\\
Les disposer sur un papier de cuisson.\\
Cuisson th 175°, 25 min pour la première fournée et que 20 pour les suivantes.\\
Préparer le mélange de sucre vanille et sucre glace. Y rouler les gâteaux chauds.\\

\recette{Macarons au chocolat}
\begin{ingredients}
\item 4 blancs d'œufs montés en neige
\item 125 g chocolat amer
\item 250 g sucre en poudre
\item 250 amandes en poudre
\item hosties ou papier pour four
\end{ingredients}
Mélanger les ingrédients et en faire des petits tas sur les hosties.
180° pendant 15 min avec un four préchauffé. En fin de cuisson, le macaron doit être juste craquant en surface et mou à l'intérieur.

\recette{Triangle aux noix:}
Préparation: 30 min	

\begin{ingredients}
\item 150 g beurre (ou margarine)
\item 150 g sucre
\item 190 g noix moulues
\item 250 g farine
\item 1 CS rhum
\item 1 œuf
\end{ingredients}

  Battre le beurre fondu avec le sucre. Ajouter les autres ingrédients avec 2 CS d'eau. Former des prismes d'environ 2 cm de coté, et les découper en tranches d'1 cm.\\
  Cuisson 20 min à 190°C.

\recette{Nids aux noisettes}
\image{0.7}{nids_aux_noisettes.png}
Préparation: 15 min
Pour 65 pièces:

\begin{ingredients}
\item 150 g sucre
\item 100 g beurre (ou margarine) 
\item 2 œufs + 1 jaune
\item 300 g farine
\item 1/2 paquet de levure
\item     noisettes entières
\end{ingredients}

  Tourner en mousse le sucre, les œufs et le beurre. \\
  Ajouter en pluie la farine et la levure préalablement mélangées.\\
  Placer sur une plaque de four des petites boulettes de pâte (2 cm de diamètre) où une noisette est plantée au centre.\\
  Cuisson 15 min à 180°C.



\recette{Galettes à l'orange:}
Préparation: 20 min + 1h de repos
Cuisson: 12 min

Pour 40 pièces:	
\begin{ingredients}					
\item 120 g beurre (ou margarine)				
\item 2 jaunes d'œufs					  
\item 80 g sucre
\item 1/2 jus d'orange + zeste
\item 1 pointe de couteau de levure chimique
\item 250 g farine
\end{ingredients}
\textbf{Glaçage:}
\begin{ingredients}
\item 150 g sucre glace
\item 1 CS jus d'orange
\end{ingredients}

  Blanchir le mélange sucre + beurre. Ajouter les jaunes d'œuf, le jus d'orange et le zeste, puis + farine et levure. Former une boule mise a reposer 1h au frigo.\\
  Faire une abaisse de 5 mm et découper des disques (5 cm de diamètre). \\
  Cuisson 12 min à 200°.\\
  Glacer encore chaud.  

\recette{Petits rochers à la noix de coco:}
\image{0.5}{rochers_coco.png}
  Pour 30 pièces :
  \begin{ingredients}
  \item 200 g noix de coco râpée
  \item 120 g sucre poudre (sucre déduit)
	\item 2 œufs
   \item 40 g beurre
  \end{ingredients}
  
	Noix de coco + sucre + beurre. Faire fondre le beurre 30 secondes au micro-onde.\\
  + 2 œufs, mélanger rapidement.\\
	 Disposer de petits tas façonnés en forme de pyramide sur une plaque de cuisson. Cuisson 10 min à 180°.


\recette{Croquants aux noix}
\image{0.5}{croquants_noix.png}
\begin{ingredients}
\item 1 tasse de noix moulues
\item 1/2 tasse de crème fraîche
\item 3/4 tasse de sucre (sucre réduit)
\item 3/4 tasse de farine
\item 1 œuf
\item cannelle
\end{ingredients}

Cuisson environ 15 min à 180°C.

\recette{Mini-cakes}
Préparation : 20 min
Cuisson : 2*15 min

Pour 48 pièces :
\begin{ingredients}
\item 190 g farine
\item 190 g sucre
\item 150 g beurre
\item 4 œufs
\item 1/2 sachet de levure
\item sel
\end{ingredients}

  Dans un saladier, mélanger le sucre avec les œufs et le sel pour obtenir une préparation mousseuse. + farine, + levure + beurre ramolli.\\
  Préchauffer le four à 180°C.\\
  Beurrer la plaque a mini-cakes et verser-y la pâte en remplissant entre la moitié et les 3/4.\\
  Cuisson 15-20 min. Démouler à froid.\\
\\
Variante : ces gâteaux peuvent être parfumés avec zeste d’orange, de citron, pépites de chocolats, raisins secs ...

\recette{Croquets}
\image{0.5}{croquets.png}
Pour la préparation de 2 pains lors de la cuisson :
\begin{ingredients}
\item 500 g farine
\item 4 œufs
\item 250 g sucre poudre
\item 70 g huile
\item 1 sachet de levure chimique
\item 1 CS rhum
\item 200 g amandes ou noisettes entières
\end{ingredients}

  Mélanger le tout en n’ajoutant pas la levure en dernier afin de mieux la répartir.\\
  Constituer 2 pains de 1 à 2 cm d’épaisseur. Les disposer sur une plaque à four farinée.\\
  Cuisson 40 min à 180°C.\\
  A la sortie du four, découper les pains en tranches de 1 cm.\\
  Laisser refroidir. 

\recette{Petits beurre aux fruits confits :}
\image{0.5}{Petits_beurre_aux_fruits_confits.png}
Pour 40-50 biscuits :
\begin{ingredients}
\item 180 g beurre
\item 200 g cassonade
\item 100 g fruits confits
\item 400 g farine
\item 1 sachet sucre vanillé 
\item 1 œuf
\end{ingredients}

  Cassonade + sucre vanillé + beurre battu en crème + œuf puis farine et fruits confits.\\
  Former 1 ou 2 rouleaux de 3 cm de diamètre. \\
Envelopper de film plastique ou de papier sulfurisé pour les mettre au moins 30 min à 4°C.\\
  Couper les rouleaux en tranches de 0.5 cm et les disposer sur du papier sulfurisé.\\
  Cuisson 15 min à 180°C.\\

Variante : Remplacer les fruits confits par 100 à 200 g de raisins secs.

\recette{Croquants aux cacahouètes}
\image{0.5}{Croquants_cacahouètes.png}
\begin{ingredients}
\item 100 g cacahouètes moulues
\item 50 g farine
\item 120 g sucre
\item 140 g crème fraîche
\item 1 œuf
\item 5 CS lait
\end{ingredients}

  Mélanger tout les ingrédients. Remplir à ras les moules rectangulaires en silicone. Cuisson 20-25 min à 180°C.

\recette{Petits beurres}
  Pour 30 biscuits
\begin{ingredients}
  \item 125 g beurre
  \item 100 g sucre
  \item 1 sachet sucre vanillé
  \item 200 g farine
  \item 1 œuf
  \item 1,5 CC levure
\end{ingredients}
  
	Beurre + sucres + œuf, puis + farine et levure.\\
	Déposer 1 CC de pâte sur la plaque de cuisson.\\
	Cuisson 15 min à 180°C.\\
Se garde bien.

\end{document}